\documentclass{yqwu-note}

\title{Wolfram语言教程}
\author{Yuanqing Wu}
\department{Hongyi Honor School}
\date{\today}
\project{Program Language}

\begin{document}
  \maketitle
  \tableofcontents

  \section*{Preface}
  这{\bf 并不是}一个系统和完整的教程, 而偏向于一个集邮式的大杂烩, 其中的内容
  按照Function, Application和Topic来分类.\\
  简单来说, Function内分别讨论一个单独函数的用法,tricks和一些常见问题; 
  Application中则讨论一些常见问题的solutions及他们之间的区别和对比;
  Topic则是一些其他的内容, 例如两个相见函数的对比, 一些泛用的技巧等等.\\
  由于本人精力有限, 因此更新速度和准确性难以保证, 请谨慎参考并自行验证正确性.
  \begin{remark*}{}{}
    请记住,最好的学习Wolfram语言和Mathematica的方法永远是
    \href{https://reference.wolfram.com/language/index.html.zh}{官方文档}和F1.
  \end{remark*}
  FYI, 本教程中所有代码和结果均由
  {\bf Wolfram Language 13.0.0 Engine for Linux x86 (64-bit)}得到.

  \section{Functions}


  \section{Applications}


  \section{Topics}



\end{document}
